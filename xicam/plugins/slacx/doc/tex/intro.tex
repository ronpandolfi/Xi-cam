\section{Introduction}

The \verb|slacx| software package provides 
a fast and lean platform for processing image-like data.
It is being developed to perform analysis of x-ray diffraction patterns 
for ongoing projects at SLAC/SSRL.
At the core of \verb|slacx| is a non-graphical workflow engine
meant for batch processing based on instructions from a text input file.
On top of that engine is a graphical user interface 
for users to develop and debug their workflows.
Other features include:
\begin{itemize}
\item A plugin module for using slacx and its components 
    in the \verb|xi-cam| software package
\item An operation development interface 
    for users with little or no programming experience 
    to write their own processing routines 
\item A network client for directing remote (distributed) computations 
    and communicating with remote filesystems to store and retrieve data
\end{itemize}
Some long-term goals of \verb|slacx| are: 
\begin{itemize}
\item to streamline data analysis and standardize storage 
    to make data easily accessible for future reference
\item eliminate the development of redundant analysis routines,
    reducing the potential for bugs
    and moving towards highly refined and efficient analysis routines
\item to interface with experimental equipment
    to enable model-driven feedback 
    targeting specific engineering objectives
\end{itemize}

The \verb|slacx| developers would love to hear from you
if you have wisdom, thoughts, haikus, bugs, artwork, suggestions, or limericks.
Get in touch with us at \verb|slacx-developers@slac.stanford.edu|.




